% \iffalse meta-comment
%
% Copyright (C) 2023 by BenSt099
% -------------------------------------------------------
% 
% This file may be distributed and/or modified under the
% conditions of the LaTeX Project Public License, either version 1.3
% of this license or (at your option) any later version.
% The latest version of this license is in:
%
%    http://www.latex-project.org/lppl.txt
%
% and version 1.3 or later is part of all distributions of LaTeX 
% version 2005/12/01 or later.
%
% \fi
%
% \iffalse
%<*driver>
\ProvidesFile{isosafety.dtx}
%</driver>
%<package>\NeedsTeXFormat{LaTeX2e}[2005/12/01]
%<package>\ProvidesPackage{isosafety}
%<*package>
    [2023/11/09 v1.1 ISO Safety Colors And Safety Signs]

\RequirePackage{graphicx}
\RequirePackage{xcolor}
%</package>
%
%<*driver>
\documentclass{ltxdoc}
\usepackage{isosafety}[2023/11/09]
\usepackage{hyperref}
\EnableCrossrefs         
\CodelineIndex
\RecordChanges
\begin{document}
    \DocInput{isosafety.dtx}
    \PrintChanges
    \PrintIndex
\end{document}
%</driver>
% \fi
%
%
% \changes{v1.1}{2023/11/09}{Beta version}
%
% \GetFileInfo{isosafety.sty}
%
% \DoNotIndex{\newcommand,\newenvironment}
% 
%
% \title{The \textsf{isosafety} package\thanks{This document
%   corresponds to \textsf{isosafety}~\fileversion, dated \filedate.}}
% \author{BenSt099 \\ \texttt{github.com/BenSt099}}
%
% \maketitle
%
% \section{Introduction}
%
% This package provides ISO colors and ISO signs according to the ISO standards 3864 and 7010.
% It can be useful when creating instructions for chemical or physical experiments.
%
% \section{Usage}
%
% \subsection{ISO colors}
%
% There are 6 colors that are also used in the signs: \vspace*{5mm}
%
% \begin{tabular}{cc}
%    Warning & {\color{Warning} Warning Color} \\
%    Prohibition & {\color{Prohibition} Prohibition Color} \\
%    Mandatory & {\color{Mandatory} Mandatory Color} \\
%    Rescue & {\color{Rescue} Rescue Color} \\
%    Backgrounds & {\color{Backgrounds} Backgrounds Color} \\
%    Symbol & {\color{Symbol} Symbol Color} \\
% \end{tabular}
%
% \subsection{Colors - Example}
%
% \begin{verbatim}
%    {\color{Mandatory} Text with mandatory color}
% \end{verbatim}
%
% \subsection{ISO signs}
%
% The signs from ISO are placed in a category, have a letter and a number
%
% \begin{itemize}
%    \item Safe condition - Letter E (001 - 070)
%    \item Crescent variant - CV (003, 004, 009, 010, 011, 012, 013, 027, 028, 029, 064, 067)
%    \item Fire Protection - Letter F (001 - 019)
%    \item Mandatory - Letter M (001 - 060)
%    \item Prohibition - Letter P (001 - 074)
%    \item Warning - Letter W (001 - 080)
% \end{itemize}
%
% The command to access these signs has the following syntax:
%
% \begin{verbatim}
%    \Isosign{ISO\_7010\_{Letter}{Number}.pdf}
% \end{verbatim}
%
% \subsection{Signs - Example}
%
% Imagine you want to put the sign of a fire extinguisher in your document.
% Use the following command:
% 
% \begin{verbatim}
%    \Isosign{ISO\_7010\_F001.pdf}
% \end{verbatim}
%
% \noindent For an overview, take a look at \href{https://en.wikipedia.org/wiki/ISO_7010}{Wikipedia}
%
% \StopEventually{}

%%%%%%%%%%%%%%%%%%%%%%%%%%%%%%%%%%%%%%%%%%%%%%%

\definecolor{Warning}{HTML}{F9A800}  
\definecolor{Prohibition}{HTML}{9B2423}  
\definecolor{Mandatory}{HTML}{005387}  
\definecolor{Rescue}{HTML}{237F52}  
\definecolor{Backgrounds}{HTML}{ECECE7}  
\definecolor{Symbol}{HTML}{2B2B2C}  

%%%%%%%%%%%%%%%%%%%%%%%%%%%%%%%%%%%%%%%%%%%%%%%

%%%%%%%%%%%%%%%%%%%%%%%%%%%%%%%%%%%%%%%%%%%%%%%
\newcommand{\Isosign}[1]{isosafety-pdfs/ISO_7010_#1.pdf}
%%%%%%%%%%%%%%%%%%%%%%%%%%%%%%%%%%%%%%%%%%%%%%%

%%%%%%%%%%%%%%%%%%%%%%%%%%%%%%%%%%%%%%%%%%%%%%%
% \Finale
\endinput