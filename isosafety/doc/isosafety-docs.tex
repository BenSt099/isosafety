\documentclass{article}

\usepackage{xcolor}
\definecolor{Caution}{HTML}{F9A800}
\definecolor{Warning}{HTML}{D05D29} 
\definecolor{Prohibition}{HTML}{9B2423}  
\definecolor{Mandatory}{HTML}{005387}  
\definecolor{Rescue}{HTML}{237F52}  
\definecolor{Backgrounds}{HTML}{ECECE7}  
\definecolor{Symbol}{HTML}{2B2B2C} 
\usepackage{array}
\usepackage{hyperref}
\usepackage{verbatim}
\makeatletter
\newcommand{\verbatimfontfamily}[1]{\def\verbatim@font{#1}}%
\makeatother
\usepackage{colortbl}
\usepackage{geometry}
\geometry{
    left=2cm,
    right=2cm,
    top=3cm,
    bottom=3cm
}

\begin{document}

\begin{center}
    {\Huge isosafety} \vspace{5mm}

    {\Large Version 1.2}\vspace{5mm}

    {\Large August 20, 2024}\vspace{5mm}

    \href{https://github.com/BenSt099/isosafety}{github.com/BenSt099/isosafety} 
\end{center}\vspace{5mm}

\section{General information}

This is the official documentation of the package \textbf{isosafety}. It provides ISO colors and signs according to the ISO standards
3864 and 7010. It can be used to create instructions for chemical or physical experiments.\vspace{5mm}

\noindent{\large \textbf{NOTE}}

\noindent This is not an official package from the ISO. All signs are taken from Wikipedia.

\section{Dependencies}

This package has the following dependencies: \texttt{graphicx}, \texttt{xcolor}, \texttt{ifthen}, \texttt{xkeyval}

\section{ISO Colors}

Seven colors are defined which are also used to create the signs: \vspace{5mm}

\renewcommand{\arraystretch}{1.2}
    \begin{tabular}{ll}
        \hline
        Name & Colors \\ 
        \hline
        Caution & \cellcolor{Caution} \\
        Warning & \cellcolor{Warning} \\
        Prohibition & \cellcolor{Prohibition} \\
        Mandatory & \cellcolor{Mandatory} \\
        Rescue & \cellcolor{Rescue} \\
        Backgrounds & \cellcolor{Backgrounds} \\
        Smybol & \cellcolor{Symbol} \\
        \hline
\end{tabular}\vspace{5mm}

\noindent How to use: \vspace{3mm}

\verbatimfontfamily{\slshape\ttfamily}

\begin{verbatim}
{\color{Mandatory} \textbf{Text with Mandatory color.}}
\end{verbatim}\vspace{2mm}

\noindent{\color{Mandatory} \textbf{Text with Mandatory color.}}

\newpage

\section{ISO Signs}

\noindent This section deals with the construction of signs. To begin with, each sign is placed in a category, has a letter and a number: \vspace{5mm}

\renewcommand{\arraystretch}{1.2}
    \begin{tabular}{lll}
        \hline
        Category & Letter(s) & Numbers \\ 
        \hline
        Safe condition &  E & (001 - 070) \\
        Crescent variant & CV & (003, 004, 009, 010, 011, 012, 013, 027, 028, 029, 064, 067) \\
        Fire Protection &  F & (001 - 019) \\
        Mandatory &  M & (001 - 060) \\
        Prohibition &  P & (001 - 074) \\
        Warning & W & (001 - 080) \\
        \hline
\end{tabular}\vspace{5mm}

\noindent The command to access a sign is: \begin{verbatim}
    \Isosign{<Letter><Number>}
\end{verbatim}\vspace{5mm}

\noindent This command constructs a path to a sign in your \TeX Live / Mik\TeX / \ldots $\;$ installation. That means to actually display the sign, resize it, edit it, etc., follow the given example:

\begin{verbatim}
    %%% Example file   
    \documentclass{article}
        
                            % example path
    \usepackage[ fullpath = /texlive/2024/texmf-dist/tex/latex/isosafety ]{isosafety}

    \usepackage{graphicx}

    \begin{document}

        \includegraphics{\Isosign{F001}}
        \includegraphics[scale=2]{\Isosign{P074}}

    \end{document}
\end{verbatim}\vspace{5mm}

\noindent As you can notice, when we import the package, we provide it with an option. The option is named \textbf{fullpath} and is \textbf{necessary}!
Since you may have a different version of \TeX Live or maybe even a different \TeX-installation, you have to provide a path that leads to the installation directory 
of isosafety. The path that is given in the example is the typical path on most systems. The year \textit{2024} refers to the version of \TeX Live installed on your system \vspace{5mm}

\noindent\textbf{NOTE}: On Windows, the path should also be given with foreslashes (just like in the example). \\

\noindent\textbf{NOTE}: The last directory in the path \textit{does not} end with a foreslash (just like in the example). \\

\noindent\textbf{NOTE}: For a complete overview of the available signs, take a look at \href{https://en.wikipedia.org/wiki/ISO_7010}{Wikipedia}.






\end{document}